\section{RabbitMQ}
\index{RabbitMQ}

RabbitMQ technology is open source message broker, which supports multiple
messaging protocols. It has many features such as asynchronous messaging,
which supports message queuing, receive and deliver acknowledgments, routing
any message queues with broadcasting to logs or messages to multiple users.

On the collection of nodes which is also known as clusters, Users shares the 
many resources among several Erlan nodes, where we run RabbitMQ applications 
are running and host them virtually, with managing queues and maintain runtime
parameters. ~\cite{hid-sp18-520-RabbitMQCluster}.

Serveral authentication mechanisms are been implemented and can be customized
with RabbitMQ including SASL. By default the authentication in RabbitMQ is
very PLAIN and requires setup on respective servers and clients.
~\cite{hid-sp18-520-RabbitMQauth}.

With the advancement of the technology which was earlier implemented as
Advance Message Queuing and now it been enhanced for supporting
streaming Text orinented messaging protocol. There are many other protocols
have been developed to improve the technology.
~\cite{hid-sp18-520-RabbitMQwiki}.

A browser UI based API is provided for monitoring and managing RabbitMQ
servers. The admins can use this API for addressing and any kind of updates 
needed to provide better services.~\cite{hid-sp18-520-RabbitMQmana}.
