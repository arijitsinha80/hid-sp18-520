% status: 95
% chapter: Amazon Relational Database Services (RDS)

\title{Amazon Relational Database Services (RDS)}

\author{Arijit Sinha}
\affiliation{%
  \institution{Indiana University}
  \city{Bloomington} 
  \state{IN} 
  \postcode{47408}
  \country{USA}}
\email{arisinha@iu.edu}

% The default list of authors is too long for headers}
\renewcommand{\shortauthors}{A.Sinha}

\begin{abstract}
Amazon Relational Database Service also known as RDS is cloud 
computing platform providing a prime web service to operate 
with relational databases. With AWS database services, 
it provides an mechanism for creating/ replicating/ migrating 
any existing databases on AWS cloud.

\end{abstract}

\keywords{520, i524, Amazon RDS, RDS, AWSRDS, AWS RDS}

\maketitle

\section{Introduction}
Amazon Relational database services are the web service which 
has been effectively used for  handling and managing relational 
databases, which in return provides high performance, security, 
maximum availability and compatibility. 
It is compatible with variety of database engines running in 
background including Amazon Aurora, PostgrSQL, MySQL, MariaDB, 
Oracle and Microsoft SQL Server.

\section{Key features}
Amazon RDS provided is the latest infrastructure platform with 
updated database softwares and management tools to maintain and 
perform databases administration with security and fault tolerant 
features. It check and updates the latest patches of software and 
ensuring the database are running on latest software and hardware. 
It schedule major and minor releases of the software to keep the 
Infrastructure platform updated. It also allows to configure with 
previous DB Instance version.

Below are some popular features 
\begin{itemize}
	
\item Scaling Storage-It can automatically increase the storage once 
size or volume of the databases ins reaching its maximum capacity. It 
can also 
scale storage based on high volume data getting loaded or read 
from the database. Based on the usage trend, it can scale the database 
services. It handles the read request effectively and optimally with read 
replicas, which is to create the replica of the database. 
Also, upon usage if it is not needed, it can be deleted from Management 
console or API. With API, it can manage the same operation as possible 
with managenement console for create, start, stop, modify, failover, 
describe, authorize and add DB clusters and DB instances.

\item Offers less Administrative workload-When the services are getting 
setup, all the databases instances are configured with its respective 
database engines. Amazon provides command line and management consoles for 
easy administration of the databases.

\item Reliability-It can replicate the data to a standby instance on 
different Availability Zone using the Multi-AZ DB instance. It provides  
automates backups, user defined snapshots of the data stored 
on Amazon S3. 
In the event of failure of an hardware, it can automatically replace the 
instance. Upgrade from single AZ to multi AZ can occur with no latency or 
downtime. Once the selection is done from upgrade to Multi AZ, the snapshot 
of the instance is captured, after which a new instance is built from the 
snapshot and configuration is setup for taking or keeping the 
Multi AZ databases in sync.
~\cite{hid-sp18-520-amazonrdsfaqs}

\item High Performance and Secure-It provides high performance using General 
purpose SSD storage and Provisioned IOPS SSD storage. It provides the feature 
of encrypting the databases using keys (AWS Key Management Services). 
Along with, it provides Amazon VPC for network isolation for databases on 
cloud to securely connect with on premise applications. 
With in this Amazon VPC, we may have mutiple subnets with atleast on 
the AZ zone. Data restoration or migration outside VPN is prohibited and 
it is not supported. 

\end{itemize}

\section{Building Blocks for Amazon RDS}

DB Instances are Amazon RDS primary building blocks which is a secured 
database environment on AWS cloud. DB Instance can consists of multiple databases. 
As mentioned in above section, these DB instance can 
be easily managed using simple API, AWS management console and Command line 
interfaces to set the configurations and monitor the behavior and capabilities 
of relational databases. It does not need any additional database maintenance 
software. In these DB instances, we can have multiple databases created by many 
users or 
applications.
In the background, we have DB engines interacting with DB instances. Few of the 
examples can be ``MySQL, Maria DB, PostgreSQL, Oracle and Microsoft SQL Server DB 
engines''~\cite{hid-sp18-520-amazonrds}.
There are 3 types of storage available with DB instances (Magnetic, General 
Purpose SSD and Provisioned IOPS).
Storage capacity depends on various storage type and respective database engines 
it been configured.
Amazon RDS can select IP address range, subnets, access control list and 
configure routing to make it more secure and reliable.
Another component provided by Amazon is IAM (Identity and Access Management), 
which this you can provide provision on users to create, delete, modify read 
any DB instances.

\section{RDS Automated and Manual Monitoring Tools}

Amazon RDS can monitored for it performance and can be reported in case of 
any issues or failures on DB instances, DB clusters DB Cluster Snapshots, DB 
parameter group or DB security group.
On real time, DB instances or clusters can be monitored. It also maintains the 
database log files which can referred or consulted in cases of any failure or 
issues encountered.  
Amazon TRDS also provided extra feature for monitoring with CloudWatch for 
metrics, alarms and logs, along with service health status. 
With Command Prompt-using below command can view performance metrics and 
alarm- 

``aws cloudwatch list-metrics --namespace AWS/RDS

put-metric-alarm''~\cite{hid-sp18-520-amardsmon}

With API-using the CloudWatch API GetMetricStatistics with start and end time 
can provide detail metrics on performance and form setting up alarm 
with ``PutMetricAlarm''~\cite{hid-sp18-520-amardsmon} on DB Instance.

Based on the user defined baseline for performance and resource to be monitored 
Amazon RDS store the respective monitoring logs including your CPU, RAM, Disk 
Space consumption. It can monitor the network traffics and help in deciding 
the throughput details. With Amazon RDS, it helps in Disk space monitoring 
helps in taking decision, if the data needs to be purged or archived.
Number of users connected to database can be monitored with kind of operation 
getting performed on Database with Amazon RDS console. There are configuration
monitoring to identify the changes to configurations on DB instance using a 
service AWS Config like security, subnet, events on DB instance.

Various Amazon RDS metrics dimensions can be on name of the Engine, specific 
DB Instances, DB Clusters with Roles and database class. 
It uses EBS volumes, which it automatically adjusts to upgrade and 
enhance performance.
IOPS (SSD) storage is recommended with High workloads from Online 
transaction processing data.
General purpose (SSD) stroage is recommended for workloads with small
scale on database.
It also provides the enhanced monitoring, which will be used for monitoring 
the health of DB instance. It will help in monitoring the opererating system 
with process being executed details. It will capture the system level metrics 
for CPU, memory, file system and disk I/O.


Amazon RDS can also encryption on databases with Amazon Key management services. 
It can restore point in time data as part of recovery process.It can 
automatically initiate the failover process, if we can not access the 
primary AZ, can't connect to primary on network, failure of storage.
Database events can be integrated with another amazon service known as
Amazon SNS, which can send the SMS text messages.
~\cite{hid-sp18-520-amardsmon}.

\section{Create DB Instance-Example with MySQL}
Creating a DB instance using MySQL Database engine.
As prerequisite, we need to have access on AWS management console.
For initial DB Instance set-up, it is managed through AWS Management
Console.
Search from Database section having listing as RDS.
Navigate to open Amazon RDS Console.
Next, its needed to have the understanding on what configuration is needed 
for creating MySQL DB instance. For example, we can have certain amount of 
storage and backup strategy to be build.
From the AWS RDS Console, we first need to select the region, also 
known as Availability zone, where we can host AWS RDB activities.
From AWS RDB Console, we next need to launch DB Instance from 
Instances menu.
Next, we will provided with option to select the SQL Engine for the DB 
Instance. As we are taking the example for MySQL, we will select the 
MySQL Engine.
From Console, we need to select the Database engine named MySQL.
Need to select the purpose of the database instance like for 
production purpose, we can select the radio button as Dev/Test.
In this step, we need to specify the DB details or configurations
DB Instance Specifications with license, engine version, class, multi-AZ 
deployment, storage type, storage allocation.
Provide the Identification name of DB Instance, username and password.
DB Instance can be configured with network security-Virtual private 
cloud, security group, subnet group.

Provide the database name, port-default to 3306, DB parameter group. 
Along with backup strategy and monitoring capabilities cab be defined 
while creating.
Next launch DB Instance button on the console.

These above steps have completed the creation DB Instance.The Status- 
creating for DB Instance changes to ready for use and then to available.

Once the status is changed to available, it is ready to be connected 
with SQL client.

In this example, we are configuring MySQL, so we can download and connect 
using MySQL workbench as SQL client for MySQL database. 
On MySQL Workbench-Database, Click on connect to database. We 
can pass the connection parameters like hostname, port -default to 3306, 
Username, and password.
Once connected to the database, we can perform DDL, DML 
statements on the database. 

You can connect to Read Replica as the same way with details on endpoints. 
DDL statement can also be performed on read Replica
~\cite{hid-sp18-520-amazonrdscreatesteps}.

\section{Paid Service}
DB Instance hours, Storage per month, I/O request per month with data transfer, 
backup storage with provisioned IOPS per month are paid services.
So the services must be deleted or stopped and avoid extra billing than usage.
~\cite{hid-sp18-520-amazonrdsfaqs}

\section{Delete DB Instance-Example with MySQL}

Once logged into Amazon RDS console, navigate to Instance Actions 
and hit the delete link.
DB instances can be deleted after taking the final snapshot or it can be 
deleted with capturing final snapshot of relational database
~\cite{hid-sp18-520-amazonrdscreatesteps}.

\section{Conclusion}
Amazon RDS is provides highly optimized and high performance web services
supporting multiple type of SQL databases, providing service to easily 
customize, configure and monitor the DB activities and administration.
We have many sources of data getting generated and are gets used for 
multiple purpose of analysis, interpretation. This cause for highly complex
and high performance databases, which as a service is provided by Amazon to 
maintain a relation database on cloud. With disaster recover mechanism reduces
the risk of downtown and latency.

\begin{acks}
  The authors would like to thank Dr.~Gregor~von~Laszewski for his
  support and suggestions to write this paper.
\end{acks}


\bibliographystyle{ACM-Reference-Format}
\bibliography{report} 
