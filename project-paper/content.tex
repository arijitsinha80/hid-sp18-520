% status: 100
% chapter: SciKit Learn Algorithm

\title{ProjectProposal of REST Service Framework with SciKit Learn Algorithm}

\author{Arijit Sinha}
\affiliation{%
  \institution{Indiana University}
  \city{Bloomington} 
  \state{IN} 
  \postcode{47408}
  \country{USA}}
\email{arisinha@iu.edu}

\author{Ritesh Tandon}
\affiliation{%
  \institution{Indiana University}
  \city{Bloomington} 
  \state{IN} 
  \postcode{47408}
  \country{USA}}
\email{ritandon@iu.edu}

% The default list of authors is too long for headers}
\renewcommand{\shortauthors}{G. v. Laszewski}


\begin{abstract}
In current world, data is getting generated and stored with different storage systems. 
We need to use this data for a better understanding, analyze and can estimate the 
future scenarios with certain probability. There are many algorithms, which have 
developed and implemented for providing better accuracy on the future scenarios.
\end{abstract}

\keywords{hidsp18520, hidsp18523, REST, scikit learn, linear, decision tree, 
random forest, algorithm, Machine, learning}


\maketitle

\section{Introduction}

Scikit learn is a library created under machine learning algorithms, which uses 
different datasets gathered over years to learn and predicts future scenarios. 
Supervised and unsupervised learning can differentiation conducted for learning 
variety of dataset.

Supervised algorithms are on the dataset, which has the target variable, which 
need to be predicted or estimated. This datasets can be acted with below different 
approaches

Classification is based on the classes and the labeled data, we need to predict 
the unlabeled data.

Regression is based on the continuous variable or data, we need to predict the 
future state of data is known as regression

Unsupervised algorithms are on the dataset, where we donot see the target variable 
for prediction, we learn its behavior set of vector input variable to identify the 
clustering group of similar behavior data or sample
~\cite{sckitml}
Kaggle is a known location for different kind of datasets gathered by various 
institutes across globe.

\section{Scope of work}

Below are the 3 algorithm from Scikit learn
\begin{itemize}
\item Implement Linear Regression
\item Implement Decision Tree
\item Implement Random Forest
\end{itemize}

We have acquired the dataset from Kaggle and read the data dictionary details on 
different websites which includes below describes attributes. We have 14204 instances and 13 
attributes in the dataset, which will be spitted into Training and Test Data set. This dataset 
is available on public websites

BigMart Dataset, With this dataset, we will predict the sale price of various products based 
on the learning of historical data in the datasets using different algorithm. The dataset has 
various data with respect to
\begin{itemize}
\item Item_Fat_Content
\item Item_Identifier
\item Item_MRP
\item Item_Outlet_Sales
\item Item_Type
\item Item_Visibility
\item Item_Weight
\item Outlet_Establishment_Year
\item Outlet_Identifier
\item Outlet_Location_Type
\item Outlet_Size
\item Outlet_Type
\item source
\end{itemize}
~\cite{kaggleds}

\section{Reason}

We are planning to use Regression learning algorithm because the target variable is 
numerical and continuous in nature. We will be creating ML pipeline using linear, 
regularized linear, tree and forest learning algorithm. We will compare and evaluate 
different models based on CV and accuracy of different learning algorithm.

\section{Technology Stack}

Python will be used for Data loading, preprocessing and cleaning. Using Scikit learn 
library, we will implement varity of algorithms to conduct above process and finally 
will predict the sale price of its products.

REST services will be implemented to provide a provide prediction of price of the 
products:

\begin{itemize}
\item REST data preprocessing: It will be the service, which does the data processing 
with removal or imputing the data with mean or median. Removal of the columns which 
doesn’t any correlation with target variable
\item REST data prediction: it will be the service, which will do multiple predictions 
using multiple algorithms as below -

\item Rest API with Linear Regression – Display the outcome of product and predicted 
price
\item Rest API with Random Forest – Display the outcome of product and predicted price
\item Rest API with Decision Tree – Display the outcome of product and predicted price
\end{itemize}

Cloud technology will be Microsoft Azure.

\section{Conclusion}

With all the 3 alogorithms, it displays lot of different ways to consider the feature
and do the prediction with high accuracy.


\begin{acks}

  The authors would like to thank Dr.~Gregor~von~Laszewski for his
  support and suggestions to write this paper.

\end{acks}

\bibliographystyle{ACM-Reference-Format}
\bibliography{report} 

